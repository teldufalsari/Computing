%%% Работа с русским языком
\usepackage{cmap}					          % поиск в PDF
\usepackage[T2A]{fontenc}			      % кодировка
\usepackage[utf8]{inputenc}               % кодировка исходного текста
\usepackage[english, russian]{babel}   % локализация и переносы
\usepackage{breqn}
\usepackage{romannum}


%%% Страница 
\usepackage{extsizes} % Возможность сделать 14-й шрифт
\usepackage{geometry}  
\geometry{left=20mm,right=20mm,top=25mm,bottom=30mm} % задание полей текста


%%%  Текст (Тут вообще всё можно настроить как надо, но вообще стандарт есть стандарт)
\setlength\parindent{0pt}         % Устанавливает длину красной строки 0pt (А я не уверен, что это нужно (!))
\sloppy                                        % строго соблюдать границы текста
\linespread{1.3}                           % коэффициент межстрочного интервала
\setlength{\parskip}{0.5em}                % вертик. интервал между абзацами

\usepackage{soulutf8} % Модификаторы начертания

%%% Для формул
\usepackage{amsmath}          
\usepackage{amssymb}
\usepackage{cases}


%%% Работа с картинками
\usepackage{graphicx}                           % Для вставки рисунков
\graphicspath{{images/}{images2/}}        % папки с картинками
\setlength\fboxsep{3pt}                    % Отступ рамки \fbox{} от рисунка
\setlength\fboxrule{1pt}                    % Толщина линий рамки \fbox{}
\usepackage{wrapfig}                     % Обтекание рисунков текстом
\graphicspath{{images/}}                     % Путь к папке с картинками


%%% Графека
\usepackage{tikz}
\usepackage{graphicx}
\usetikzlibrary{angles, babel}


%%% облегчение математических обозначений

\newcommand{\brackets}[1]{\left({#1}\right)}      % автоматический размер скобочек
\newcommand{\braces}[1]{\left\{{#1}\right\}}
\newcommand{\sqbrk}[1]{\left[{#1}\right]}
